\documentclass[10pt, letterpaper]{article}

% Packages:
\usepackage[
    ignoreheadfoot, % set margins without considering header and footer
    top=1 cm, % seperation between body and page edge from the top
    bottom=1 cm, % seperation between body and page edge from the bottom
    left=1 cm, % seperation between body and page edge from the left
    right=1 cm, % seperation between body and page edge from the right
    footskip=1.0 cm, % seperation between body and footer
    % showframe % for debugging 
]{geometry} % for adjusting page geometry
\usepackage{titlesec} % for customizing section titles
\usepackage{graphicx}
\usepackage{tabularx} % for making tables with fixed width columns
\usepackage{array} % tabularx requires this
\usepackage[dvipsnames]{xcolor} % for coloring text
\definecolor{primaryColor}{RGB}{0, 0, 0} % define primary color
\usepackage{enumitem} % for customizing lists
\usepackage{fontawesome5} % for using icons
\usepackage{amsmath} % for math
\usepackage[
    pdftitle={Utsav Chaudhary's CV},
    pdfauthor={Utsav Chaudhary},
    pdfcreator={LaTeX with RenderCV},
    colorlinks=true,
    urlcolor=primaryColor
]{hyperref} % for links, metadata and bookmarks
\usepackage[pscoord]{eso-pic} % for floating text on the page
\usepackage{calc} % for calculating lengths
\usepackage{bookmark} % for bookmarks
\usepackage{lastpage} % for getting the total number of pages
\usepackage{changepage} % for one column entries (adjustwidth environment)
\usepackage{paracol} % for two and three column entries
\usepackage{ifthen} % for conditional statements
\usepackage{needspace} % for avoiding page brake right after the section title
\usepackage{iftex} % check if engine is pdflatex, xetex or luatex

% Ensure that generate pdf is machine readable/ATS parsable:
\ifPDFTeX
    \input{glyphtounicode}
    \pdfgentounicode=1
    \usepackage[T1]{fontenc}
    \usepackage[utf8]{inputenc}
    \usepackage{lmodern}
\fi

\usepackage{charter}

% Some settings:
\raggedright
\AtBeginEnvironment{adjustwidth}{\partopsep0pt} % remove space before adjustwidth environment
\pagestyle{empty} % no header or footer
\setcounter{secnumdepth}{0} % no section numbering
\setlength{\parindent}{0pt} % no indentation
\setlength{\topskip}{0pt} % no top skip
\setlength{\columnsep}{0.15cm} % set column seperation
\pagenumbering{gobble} % no page numbering

\titleformat{\section}{\needspace{4\baselineskip}\bfseries\large}{}{0pt}{}[\vspace{1pt}\titlerule]

\titlespacing{\section}{
    % left space:
    -1pt
}{
    % top space:
    0.3 cm
}{
    % bottom space:
    0.2 cm
} % section title spacing

\renewcommand\labelitemi{$\vcenter{\hbox{\small$\bullet$}}$} % custom bullet points
\newenvironment{highlights}{
    \begin{itemize}[
        topsep=0.10 cm,
        parsep=0.10 cm,
        partopsep=0pt,
        itemsep=0pt,
        leftmargin=0 cm + 10pt
    ]
}{
    \end{itemize}
} % new environment for highlights


\newenvironment{highlightsforbulletentries}{
    \begin{itemize}[
        topsep=0.10 cm,
        parsep=0.10 cm,
        partopsep=0pt,
        itemsep=0pt,
        leftmargin=10pt
    ]
}{
    \end{itemize}
} % new environment for highlights for bullet entries

\newenvironment{onecolentry}{
    \begin{adjustwidth}{
        0 cm + 0.00001 cm
    }{
        0 cm + 0.00001 cm
    }
}{
    \end{adjustwidth}
} % new environment for one column entries

\newenvironment{twocolentry}[2][]{
    \onecolentry
    \def\secondColumn{#2}
    \setcolumnwidth{\fill, 4.5 cm}
    \begin{paracol}{2}
}{
    \switchcolumn \raggedleft \secondColumn
    \end{paracol}
    \endonecolentry
} % new environment for two column entries

\newenvironment{threecolentry}[3][]{
    \onecolentry
    \def\thirdColumn{#3}
    \setcolumnwidth{, \fill, 4.5 cm}
    \begin{paracol}{3}
    {\raggedright #2} \switchcolumn
}{
    \switchcolumn \raggedleft \thirdColumn
    \end{paracol}
    \endonecolentry
} % new environment for three column entries

\newenvironment{header}{
    \setlength{\topsep}{0pt}\par\kern\topsep\centering\linespread{1.5}
}{
    \par\kern\topsep
} % new environment for the header

\newcommand{\placelastupdatedtext}{% \placetextbox{<horizontal pos>}{<vertical pos>}{<stuff>}
  \AddToShipoutPictureFG*{% Add <stuff> to current page foreground
    \put(
        \LenToUnit{\paperwidth-2 cm-0 cm+0.05cm},
        \LenToUnit{\paperheight-1.0 cm}
    ){\vtop{{\null}\makebox[0pt][c]{
        \small\color{gray}\textit{Last updated in September 2024}\hspace{\widthof{Last updated in September 2024}}
    }}}%
  }%
}%

% save the original href command in a new command:
\let\hrefWithoutArrow\href

% new command for external links:


\begin{document}
\newcommand{\AND}{\unskip
	\cleaders\copy\ANDbox\hskip\wd\ANDbox
	\ignorespaces
}
\newsavebox\ANDbox
\sbox\ANDbox{$|$}

\begin{header}
	\fontsize{25 pt}{25 pt}\selectfont UTSAV V CHAUDHARY

	\vspace{1 pt}

	\normalsize
	\kern 5.0 pt%
	\mbox{\hrefWithoutArrow{mailto:utsavmaan28@gmail.com}{utsavmaan28@gmail.com}}%
	\kern 5.0 pt%
	\AND%
	\kern 5.0 pt%
	\mbox{\hrefWithoutArrow{https://thatinsaneguy28.netlify.app/}{portfolio}}%
	\kern 5.0 pt%
	\AND%
	\kern 5.0 pt%
	\mbox{\hrefWithoutArrow{https://github.com/UttU28/}{github/UttU28}}%
\end{header}

\vspace{2 pt - 0.3 cm}

\begin{onecolentry}
	\section{Cover Letter}
	\hspace{1cm}I'm excited to introduce myself and express my genuine \textbf{enthusiasm} for the opportunity to join your team. With a \textbf{strong passion} for problem-solving and creating meaningful solutions, I approach every challenge with a \textbf{fresh perspective} and a \textbf{commitment to excellence}. My \textbf{expertise} includes software development, web and app creation, backend engineering, and working with advanced technologies like large language models (LLMs).

	\vspace{0.15cm}
	\hspace{1cm}I love working on personal projects that challenge the status quo and allow me to approach problems from a \textbf{fresh perspective}. Whether it's building an AI-powered home assistant or creating automation tools to simplify everyday tasks, I enjoy diving into ideas that blend \textbf{innovation} with \textbf{practicality}. These projects are not only opportunities for technical growth but also a reflection of my \textbf{passion} for creativity and learning.

	\vspace{0.15cm}
	\hspace{1cm}Beyond my technical skills, I take \textbf{pride} in taking on \textbf{leadership roles} and fostering \textbf{collaboration} within teams. I approach my work with \textbf{confidence}, knowing that the effort I invest will lead to meaningful results. From brainstorming ideas to implementing solutions, I \textbf{thrive} on bringing people together and driving projects forward with clarity and purpose.

	\vspace{0.15cm}
	\hspace{1cm}I'm a \textbf{quick learner} who thrives on turning challenges into opportunities to grow and evolve. Each project I've worked on has taught me invaluable lessons, shaping my \textbf{ability} to tackle complex problems with \textbf{creativity} and \textbf{determination}. I'm excited about the opportunity to bring this mindset to your team and explore how we can innovate and grow together.

	\vspace{0.15cm}
	\hspace{1cm}Thank you for considering my application. I look forward to the chance to discuss how my skills and experiences align with your goals. Let's connect to explore the possibilities!
\end{onecolentry}

\vspace{0.3cm}

\section{Projects}
\vspace{0.05 cm}

\begin{onecolentry}
	\textbf{Apeksha (Personal Assistant + Home Assistant)} \hfill Personal Project \ \ \ \ \textbf{GitHub:} \href{https://github.com/UttU28/Apeksha}{\texttt{github/UttU28/Apeksha}}
	\begin{highlights}
		\item Fine-tuned and hosted a custom \textbf{LLM (Llama 3.2 Vision)} locally with \textbf{GPU acceleration}, ensuring high-performance functionality while keeping all data secure and local.
		\item Configured \textbf{Ollama} as the primary \textbf{Conversation Agent}, routing queries and commands to the locally running \textbf{LLM}, with unrelated queries seamlessly forwarded to external LLMs via API.
		\item Built a personal home assistant on \textbf{Raspberry Pi}, integrating intent recognition, task automation, and voice control, powered by the custom LLM and external APIs.
		\item Trained \& implemented custom wake-word detection model using \textbf{OpenWakeWord} \& \textbf{TensorFlow} for real-time detection.
		\item Deployed containers with \textbf{Wyoming Satellite} to integrate \textbf{Whisper (Speech-to-Text)} and \textbf{Piper (Text-to-Speech)} modules, enabling seamless voice interaction capabilities.
		\item Designed a \textbf{web dashboard} showcasing Apeksha's capabilities, featuring integrations with custom automation projects (available on GitHub):
		\begin{highlights}
			\item \textbf{Movie Controller}: Remote media control system using PC and smart devices. \href{https://github.com/UttU28/Movie_Controller_2}{\textcolor{blue}{\texttt{link}}}
			\item \textbf{Job Application Helper}: Automated job search and application tracking. \href{https://github.com/UttU28/Job-Application-Helper}{\textcolor{blue}{\texttt{link}}}
			\item \textbf{Bhashini}: API-based translation and recognition of 22+ Indian languages. \href{https://github.com/UttU28/Apeksha_Frontend/}{\textcolor{blue}{\texttt{link}}}
			\item \textbf{Ashwathama}: Car control system utilizing OBD2 sensor data for real-time monitoring and analytics. \href{https://github.com/UttU28/Apeksha_Frontend/}{\textcolor{blue}{\texttt{link}}}
		\end{highlights}

		\item Integrated \textbf{Firebase Database} for \textbf{authentication} and storing data related to \textbf{To-Do tasks}, \textbf{Ashwathama metrics}, and other project features. As this is a \textbf{personal project}, no external user data is stored, ensuring privacy and security.
		\item Architected a unified API-based communication framework to connect all applications, enabling cross-project voice automation and control with AI intelligence.
		\item Configured automation workflows to recognize intents and trigger actions, enhancing usability and convenience.
		\item Actively expanding features and functionality, with ongoing updates to enhance the project's capabilities and user experience.
	\end{highlights}
\end{onecolentry}

\vspace{0.15 cm}

\begin{onecolentry}
	\textbf{LinkedIn Reverse Search} \hfill Client Project
	\begin{highlights}
		\item Developed a LinkedIn scraping bot using \textbf{Python}, \textbf{BS4}, \textbf{Selenium}, and \textbf{Requests} to automate \textbf{data extraction} and validation from diverse web sources.
		\item Fine-tuned a \textbf{Hugging Face model} to detect features frm messy \textbf{HTML structures}, enabling accurate scraping, \& validation.
		\item Designed and implemented a pipeline to process input Excel sheets with headers (\texttt{firstName}, \texttt{lastName}, \texttt{companyName}) and output enriched data with additional details, including \texttt{LinkedIn URLs} and verified \texttt{Contact Emails}.
		\item Built a \textbf{threaded data enrichment pipeline} to process data in parallel across LinkedIn, SalesQL APIs, and the fine-tuned \textbf{LLM}, improving efficiency by \textbf{30\%}.
		\item Developed a client web app using \textbf{Next.js}, integrating user auth with \textbf{Firebase Auth} \& data storage via \textbf{Firestore DB}.
		\item Maintained a database of prev data to eliminate redundancies, reducing processing time for recurring requests by \textbf{40\%}.
		\item Set up a secure \textbf{SMTP server} for automated email delivery of Excel sheets to recipients, ensuring reliable data sharing.
	\end{highlights}
\end{onecolentry}


\vspace{0.15 cm}

\begin{onecolentry}
	\textbf{Eventbrite \& LUMA Scraping} \hfill Client Project
	\begin{highlights}
		\item Developed a Python bot to scrape events from \textbf{Eventbrite} and \textbf{LUMA platforms} hourly, extracting key details such as \texttt{Event Name}, \texttt{Description}, \texttt{Start Date}, \texttt{End Date}, and \texttt{Registration URL}.
		\item Designed an optimized \textbf{SQL schema} to store and manage structured event data, ensuring seamless retrieval and scalability.
		\item Built a dynamic \textbf{keyword filtering system} to identify events related to \textbf{Cryptocurrency}, \textbf{RWA}, and \textbf{Tokenization}, with additional filters for \textbf{Date Time} and \textbf{Location}.
		\item Automated \textbf{data cleaning and preprocessing} pipelines in Python to ensure relevant and accurate event information.
		\item Created a \textbf{web application} to display event data, incorporating \textbf{CRM functionalities} for tracking and managing events.
		\item Implemented a robust ingestion pipeline to handle continuous updates from event sources, improving processing speed \textbf{35\%}.
		\item Delivered tailored solutions for \textbf{keyword-based event categorization}, aligning with client-specific business goals and improving event relevance.
	\end{highlights}
\end{onecolentry}

\vspace{0.15 cm}

\begin{onecolentry}
	\textbf{Instagram Teaching Reels' Automation} \hfill Personal Project \ \ \ \ \textbf{Instagram:} \href{https://www.instagram.com/that_vocab_girl/}{\texttt{instagram/that\_vocab\_girl/}}
	\begin{highlights}
		\item Developed a Python-based pipeline to dynamically create and upload English teaching reels on Instagram, showcasing vocabulary lessons and language tips.
		\item Automated raw video collection using \textbf{PlayPhrase.me API}, storing initial files on \textbf{Google Drive} for free \& scalable storage.
		\item Utilized \textbf{FFMPEG} for advanced video editing, including transitions, overlays, and image insertions, combined with \textbf{Photoshop} for logo and design customizations.
		\item Integrated \textbf{WhisperAI} for generating subtitles with precise timestamps, enhancing accessibility \& viewer engagement.
		\item Stored and managed processed video URLs on \textbf{Azure Blob Storage}, ensuring efficient and scalable media handling.
		\item Orchestrated end-to-end workflow with \textbf{Azure DevOps} \& \textbf{CRON jobs}, achieving automated pipeline from creation to upload.
		\item Automated Instagram uploads using \textbf{Selenium} in \textbf{Docker containers} on \textbf{Ubuntu OS}, with session persistence to avoid repeated logins and added a preview dashboard for content review.
	\end{highlights}
\end{onecolentry}


\vspace{0.15 cm}

\begin{onecolentry}
	\textbf{Movie Controller Application} \hfill Personal Project \ \ \ \ \textbf{GitHub:} \href{https://github.com/UttU28/Movie_Controller_2}{\texttt{github/UttU28/Movie\_Controller\_2}}
	\begin{highlights}
		\item Developed a cross-platform media controller as a Python-based client-server application, enabling remote control of a PC for streaming platforms like \textbf{YouTube}, \textbf{Netflix}, and \textbf{Prime Video}.
		\item Designed a Smart TV-style remote interface using \textbf{Vite} and \textbf{React}, ensuring an intuitive and responsive user experience.
		\item Configured \textbf{POST requests} from \textbf{React client} to \textbf{Flask} backend, enabling seamless communication for command execution.
		\item Integrated \textbf{PyAutoGUI} to emulate PC controls like typing, searching, \& navigating through browser and media like Netflix, Prime, YouTube, and other UNdisclosed sites :) for media and content control on device.
		\item Utilized \textbf{OpenCV (CV2)} for image recognition to detect screen objects and perform targeted actions, including ad skipping and automated playback adjustments.
		\item Enabled browser-based remote access to the React client, allowing to control PC from any device within network securely.
		\item Provided a streamlined media control experience, enabling users to interact with streaming platforms through their \textbf{personal devices}, improving convenience and accessibility.
	\end{highlights}
\end{onecolentry}

\begin{onecolentry}
	\textbf{AssignmentX} \hfill Personal Project \ \ \ \ \textbf{GitHub/YouTube:} \href{https://github.com/UttU28/AssignmentX}{\texttt{github/UttU28/AssignmentX}}
	\begin{highlights}
		\item Designed \& developed an Android \& web application that generates handwritten-like assignments, utilizing \textbf{Python}, \textbf{NumPy}, \textbf{Pillow}, \textbf{Django}, \& \textbf{Android Studio}. Achieved over \textbf{5000+ downloads} \& maintained \textbf{200+ daily active users (DAUs)}.
		\item Leveraged \textbf{Pillow} and \textbf{OpenCV2} for \textbf{OCR-based image processing}, incorporating a custom \textbf{human behavioral algorithm} to replicate natural handwriting variations with realistic stroke simulation.
		\item Developed and deployed a scalable \textbf{Python RESTful API} backend on \textbf{Azure}, integrating \textbf{Azure App Services}, \textbf{Azure SQL Database}, and \textbf{Blob Storage} for seamless performance and data management.
		\item Integrated \textbf{Django} to facilitate seamless interaction between mobile, web, and backend services, ensuring a unified user experience across platforms.
		\item Enabled real-time email functionality using \textbf{SMTP}, allowing users to generate and send dynamically created PDFs of assignments directly through the app.
		\item Optimized performance using \textbf{multi-threading} \& caching techniques, reducing response times \& improving user satisfaction.
		\item Ensured a robust and scalable architecture, maintaining balance between high performance and ease of use for all platforms.
	\end{highlights}
\end{onecolentry}



% \section{Technologies}
% \begin{onecolentry}
%   \textbf{Languages:} Python, JavaScript, Java, TypeScript, C++, C\#, Rust, Bash, PowerShell, R, Go, Ruby, Swift
% \end{onecolentry}
% \begin{onecolentry}
%   \textbf{Web Technologies:} React, Flask, NextJS, NodeJS, Django, FastAPI, Express, ASP.NET, Angular, HTML, CSS, Jinja, YAML
% \end{onecolentry}
% \begin{onecolentry}
%   \textbf{Databases:} SQL, PostgreSQL, Azure SQL, Redis, AWS RDS, MongoDB, DynamoDB, Firebase, Firestore
% \end{onecolentry}
% \begin{onecolentry}
%   \textbf{Cloud \& DevOps:} Azure, AWS, CI/CD, Kubernetes, Docker, Jenkins, Terraform, Ansible, PowerShell, Yarn, NPM, Azure Functions VM Blob, AWS Lambda S3 EC2, RBAC.
% \end{onecolentry}
% \begin{onecolentry}
%   \textbf{Operating Systems:} Windows, Linux (CentOS, Ubuntu), macOS, Embedded Systems, RaspberryPi, Arduino
% \end{onecolentry}
% \begin{onecolentry}
%   \textbf{APIs and Protocols:} REST APIs, OpenAPI, Swagger, WebRTC, SOAP, GraphQL, MQTT, WebSocket, OAuth, JSON-RPC
% \end{onecolentry}
% \begin{onecolentry}
%   \textbf{Other Tools \& Technologies:} Kafka, Jira, GitHub, Nginx, VSCode, Azure AKS, Metamask, Fireblocks, Cron, WebSockets
% \end{onecolentry}



\end{document}